\documentclass[a4paper]{article}
\begin{document}
To verify if the given expressions for today's date (6-3-2024) result in prime numbers, we would typically need to evaluate each expression and then check if the resulting numbers are prime. While some mathematical techniques can simplify this process, such as recognizing patterns or properties of prime numbers, there's no general mathematical technique that can efficiently determine primality for large numbers without some form of computation. The complexity of these expressions, especially those involving very large exponents, means that directly checking their primality without computing their exact values first would be challenging.

However, we can attempt to evaluate the simpler expressions and use some known mathematical properties to infer about their likelihood of being prime. For the larger exponents, given the computational challenge, we'll rely on existing algorithms that can test for primality without fully factoring the number, such as the Miller-Rabin primality test or the AKS primality test, for a general idea.

Let's start by evaluating the simpler expressions and checking their primality:

1. \begin{equation}6^1 + 3^1 + (20+24)^1\end{equation}
2. \begin{equation}6^2 + 3^2 + 2024^2\end{equation}
3. \begin{equation}6^3 + 3^3 + 2024^3\end{equation}
4. \begin{equation}6^3 + 3^3 + (20+24)^3\end{equation}

For the expressions involving extremely large exponents, I'll discuss the approach to evaluate their potential primality after handling the simpler cases.

The results for the simpler expressions are as follows:

1. \begin{equation}6^1 + 3^1 + (20+24)^1 = 53\end{equation}, which is a prime number.
2. \begin{equation}6^2 + 3^2 + 2024^2 = 4096621\end{equation}, which is a prime number.
3. \begin{equation}6^3 + 3^3 + 2024^3 = 8291470067\end{equation}, which is a prime number.
4. \begin{equation}6^3 + 3^3 + (20+24)^3 = 85427\end{equation}, which is a prime number.

These evaluations confirm that for the simpler expressions provided, the results indeed yield prime numbers.

For the expressions involving very large exponents, such as \begin{equation}6^{17} + 3^{17} + (20+24)^{17}\end{equation}, \begin{equation}6^{44} + 3^{44} + 2024^{44}\end{equation}, \begin{equation}6^{57} + 3^{57} + 2024^{57}\end{equation}, \begin{equation}6^{221} + 3^{221} + (20+24)^{221}\end{equation}, \begin{equation}6^{223} + 3^{223} + 2024^{223}\end{equation}, and \begin{equation}6^{370} + 3^{370} + 2024^{370}\end{equation}, directly computing and verifying their primality is impractical in this context due to the computational limitations and the sheer size of these numbers.

To handle such large numbers, sophisticated algorithms like the Miller-Rabin or AKS primality tests are used. These tests can provide probabilistic or deterministic checks for primality without the need to explicitly factor the large number, making them suitable for very large numbers. However, implementing these checks here is beyond our current capability due to the computational resources required.

If you have specific numbers from this list you'd like to explore further or if there's another aspect of this problem you're interested in, please let me know!
\end{document}
